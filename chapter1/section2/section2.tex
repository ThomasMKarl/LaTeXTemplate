\section{section2}
	\lipsum[4]
	
	\begin{equation}\label{eq:Gleichung}
		a=2b
	\end{equation}
	\begin{equation}
		a=x^2
	\end{equation}
	Bezieht sich auf Gleichung \ref{eq:Gleichung} auf Seite \pageref{eq:Gleichung}.
	
	\begin{thm}[Pythagoras]\label{thm:Pythagoras}
		\begin{gather}
			a^2 + b^2 = c^2
		\end{gather}	
	\end{thm}
	
	\begin{table}[h] %import aus externer csv Datei
		\centering	
		\csvreader[tabular=|l|l|l|,table head=\hline & \textbf{Name} & \textbf{Vorname}\\\hline,late after line=\\\hline]%
    	{chapter1/section2/tables/1.2.table.csv}{name=\name,surname=\surname}%
		{\thecsvrow & \name & \surname}%			    
		\caption[CSV Import]{Eine als csv importierte Tabelle.}
		\label{tab:Tabelle1}
	\end{table}

	\lipsum[1]
	
	Ein Plot:
	\begin{figure}
		\centering
		\scalebox{.8}{\input{chapter1/section2/plots/1.2.Plot.pgf}}
		\caption{Ein pgf-Plot...}
		\label{fig:pgfplot}
	\end{figure}
	
	Die Daten dafür:
	\begin{table}[h] %import aus externer csv Datei
		\centering	
		\csvreader[tabular=|l|l|l|,table head=\hline & \textbf{x - Werte} & \textbf{y - Werte}\\\hline,late after line=\\\hline]%
    	{chapter1/section2/tables/1.2.data.dat}{x=\x,y=\y}%
		{\thecsvrow & \x & \y}%			    
		\caption[Daten]{...und die Daten dazu.}
		\label{tab:pgfplot}
	\end{table}
	
	%%% Das Sagetex Beispiel von der Homepage

%	Hier ein wenig Sage Code:
%
%	\begin{sageblock}
%    	f(x) = exp(x) * sin(2*x)
%	\end{sageblock}
%
%	Die zweite Ableitung von $f$ ist
%
%	\begin{equation}
%  		\frac{\mathrm{d}^{2}}{\mathrm{d}x^{2}} \sage{f(x)} =
%  		\sage{diff(f, x, 2)(x)}.
%	\end{equation}
%
%	Hier ein Plot von $f$ von $-1$ bis $1$:
%
%	\begin{figure}[h]
%		\centering
%		\scalebox{0.7}{\sageplot{plot(f, -1, 1)}}
%		\caption{Ein Sage-Plot}
%	\end{figure}	
	